\chapter{Results}

This chapter details the primary findings from the experiments conducted to assess the impact of soft-prompt tuning on biases related to sexual identity and orientation in large language models (LLMs). It provides a comprehensive statistical analysis of the model's performance before and after the implementation of soft-prompt tuning, highlighting significant changes in how the model processes and generates content related to sexual identity. The findings are structured to offer clear insights into the effectiveness of soft-prompt tuning as a method for bias mitigation in AI, with the use of tables, charts, and visualization plots to facilitate understanding.


\section{Hypothesis}
Hypotheses are put forward below to be tested through quantitative and qualitative evaluation and analysis of the final \textit{fined-tuned} model via soft-prompt tuning.

\renewcommand{\labelenumi}{\textbf{\Roman{enumi}.}}
\begin{enumerate}
\item \textbf{\acrshort{llms} inherit bias in relation to Gender and Sexual Identity}
\par \bigskip
\item \textbf{Soft-prompt tuning can keep up with normal fine-tuning}
\par \bigskip
\item \textbf{Gender and Sexual Identity bias can be reduced via soft-prompt tuning with using carefully selected datasets}
\par \bigskip

\end{enumerate}

\section{Overview of Experimental Findings}

This section summarizes the key outcomes of the experimentation, setting the stage for a detailed exploration of the results. It provides a high-level overview of the observed changes in the model's behavior, preparing the reader for the in-depth analysis that follows.

\section{Quantitative Analysis of Model Performance}

\subsection{Performance Metrics}

Details on the specific metrics used to evaluate the model's performance pre- and post-soft-prompt tuning are presented. This might include accuracy, fairness metrics, and other relevant measures that illuminate the model's handling of content related to sexual identity.

\subsection{Statistical Analysis}

A thorough statistical analysis is provided, comparing the performance metrics of the model before and after the application of soft-prompt tuning. This section may include t-tests, ANOVA, or other statistical methods to assess the significance of the observed changes.

\section{Qualitative Analysis of Model Outputs}

\subsection{Content Analysis}

This part delves into a qualitative evaluation of the model's outputs, examining how the representations of sexual identity and orientation have shifted following the implementation of soft-prompt tuning. Examples of model-generated text before and after tuning could be showcased to highlight specific changes.

\subsection{Bias Mitigation Assessment}

An assessment of the effectiveness of soft-prompt tuning in mitigating biases related to sexual identity. This includes a discussion on the reduction of harmful stereotypes and the improvement in the model's inclusivity and fairness.

\section{Visual Representation of Findings}

\subsection{Tables of Metrics}

Tables summarizing the performance metrics and statistical analysis results provide a clear, concise view of the quantitative changes resulting from soft-prompt tuning.

\subsection{Visualization Plots}

Pre- and Post-Tuning Comparison Plots: Graphs illustrating the comparison of model performance metrics before and after soft-prompt tuning.
Bias Distribution Graphs: Visualization of the distribution of biased versus unbiased outputs in the model's responses, highlighting the impact of the tuning process.
Content Analysis Visuals: Charts or word clouds derived from the qualitative content analysis, showcasing the differences in how sexual identity is represented.
\section{Interpretation of Results}

This section interprets the implications of the findings, discussing the observed changes in the context of bias mitigation efforts. It evaluates the success of soft-prompt tuning in addressing biases related to sexual identity and orientation within the model and considers the broader implications for the development of more ethical and fair AI systems.

\section{Summary of Results}

A concise summary of the chapter, encapsulating the major findings and their significance in the pursuit of reducing bias in AI through soft-prompt tuning. This summary sets the stage for the subsequent discussion on the implications, limitations, and potential future directions of this research.