\chapter{Methodology}

This chapter delineates the methodological framework adopted in this thesis to investigate the latent effects of biases related to sexual identity and orientation in generative models, focusing specifically on the application of soft-prompt tuning in \acrshort{llms}. The methodology is designed to critically evaluate the presence and extent of biases and to assess the effectiveness of soft-prompt tuning as a fine-tuning and bias mitigation strategy. Both quantitative and qualitative analyses are applied, to provide a comprehensive understanding of biases within \acrshort{llms} and the impact of fine-tuning alternatives and mitigation techniques.

\section{Research Design}

The study employs an exploratory research design, aiming to uncover the nature and extent of biases in LLMs and to investigate the potential of soft-prompt tuning for reducing these biases. This design facilitates a deep dive into both the detection of biases and the evaluation of soft-prompt tuning's effectiveness, thereby contributing to the development of more equitable AI systems.

\subsection{Model Choice}
For this study, the choice of the LLM was critical, as it needed to sufficiently represent the state-of-the-art capabilities while being amenable to modifications like soft-prompt tuning. After careful consideration, the model selected for this research is Mistral, a robust and well-documented LLM known for its extensive use in both academic and industrial settings. Add model details also.


\section{Bias Detection Methods}

\subsection{Quantitative Analysis}

Quantitative methods are employed to detect and measure biases in the model's output before and after the application of soft-prompt tuning. This includes statistical analysis of language patterns and the frequency of biased versus unbiased representations related to sexual identity and orientation.

\subsection{Qualitative Analysis}

Qualitative methods complement the quantitative analysis by examining the context and subtleties of the model's output. This involves content analysis to understand how sexual identity is represented and whether soft-prompt tuning influences the portrayal of diverse identities.

\section{Soft-Prompt Tuning}

\subsection{Rationale and Implementation}

Soft-prompt tuning is chosen for its efficiency and minimal impact on the underlying model structure, making it a potentially powerful tool for bias mitigation without the need for extensive retraining. The implementation involves the development of specific prompts that signal the model to adjust its responses in a way that minimizes biased outputs. The technical specifics of the soft-prompt tuning process, including the selection of prompts and the adjustment of model parameters, are detailed to provide clarity on the methodology's application.

\subsection{Advantages}

The advantages of soft-prompt tuning over traditional fine-tuning methods are discussed, emphasizing its resource efficiency and the reduced risk of compromising the model's general utility. This section justifies the selection of soft-prompt tuning as a methodologically sound approach to bias mitigation in the context of this research.


\section{Dataset Description}

\subsection{Selection and Preprocessing}

\textcolor{bhtBlue}{The datasets used for training and testing, including HolisticBias and WinoQueer datasets, are described. The selection criteria, preprocessing steps, and ethical considerations in using these datasets are discussed to ensure the research's integrity and respect for represented identities.}

For evaluation and identification of bias in a \acrshort{llm}, the WinoQueer \citep{winoqueer} Dataset is modified and used. The original dataset contains only the terms "Straight", "Heterosexual", "Cisgender" and "Cis" with the counterparts of "LGBTQ", "Queer", "Transgender", "Bisexual", "Pansexual", "Lesbian", Asexual", "Gay" and "NB". Gender and Sexual Identity target groups are mixed. As these terms do not fall into the same group, they will be separated. Originally, each of the \textit{heteronormative/binary} terms are combined with every other term. 
Therefor two groups, one for Gender and one for Sexual Identity are created. Within these groups, each term from the \textit{heteronormative} or \textit{binary} view is combined with a term outside from that.
Additionally, more terms as well as alternatives are added. Concluding in a dataset containing the following two groups:  

\paragraph{Gender Identity:}
\begin{itemize}
    \item cis, cisgender
    \item non-binary, enby, trans*, transgender, inter*, intersex, gender-fluid
\end{itemize}

\paragraph{Sexual Identity:}
\begin{itemize}
    \item heterosexual, hetero, straight
    \item bisexual, bi, homosexual, homo, gay, lesbian, queer, pansexual, pan, asexual, ace, demisexual, demi
\end{itemize}


\subsection{Application in Soft-Prompt Tuning}

This section details how the datasets are specifically applied within the soft-prompt tuning process, including the role they play in identifying bias and evaluating the effectiveness of tuning efforts.


For soft-prompt tuning vocab initialization will be chosen as it outperforms random initialization when using smaller models.

\section{Statistical Methods for Data Analysis}

The statistical methods employed to analyze the data and measure the effectiveness of soft-prompt tuning in mitigating biases are outlined. This includes the use of specific metrics for bias detection and the statistical tests applied to compare pre- and post-tuning outputs.

\section{Ethical Considerations}

The ethical implications of conducting research on biases related to sexual identity in AI are addressed. This includes considerations related to the respectful representation of diverse identities and the potential impacts of the findings on the development of equitable AI technologies.

This chapter sets the stage for a rigorous investigation into biases within LLMs and the exploration of soft-prompt tuning as a novel approach to bias mitigation. By outlining the research design, methodology, and ethical considerations, it provides a clear framework for the study's execution and the subsequent analysis of findings.